\documentclass[12pt,a4paper]{article}

%%%%%%%%%%%%%%%引入Package%%%%%%%%%%%%%%%
\usepackage[margin=3cm]{geometry} % 上下左右距離邊緣2cm
\usepackage{amssymb,amsmath,amsthm} % 引入 AMS 數學環境
\usepackage{yhmath}      % math symbol
\usepackage{mathtools}
\usepackage{mathtools}
\usepackage{graphicx}    % 圖形插入用
%\graphicspath{{images/}}  % 搜尋圖片目錄
\usepackage[normalem]{ulem} %strikethrough
%\usepackage{wrapfig}     % 文繞圖
%\usepackage{floatflt}    % 浮動 figure
%\usepackage{float}       % 浮動環境
%\usepackage{subfig}      % subfigures
%\usepackage{caption3}    % caption 增強
%\usepackage{setspace}    % 控制空行
\usepackage{fontspec}    % 加這個就可以設定字體
\usepackage{type1cm}	 % 設定fontsize用
\usepackage{titlesec}   % 設定section等的字體
\usepackage{titling}    % 加強 title 功能

\usepackage{fancyhdr}   % 頁首頁尾
\usepackage{tabularx}   % 加強版 table
\usepackage[square, comma, numbers, super, sort&compress]{natbib}
% cite加強版
\usepackage[unicode=true, pdfborder={0 0 0}, bookmarksdepth=-1]{hyperref}
% ref加強版
%\usepackage{soul}       % highlight
%\usepackage{ulem}       % 字加裝飾
\usepackage[usenames, dvipsnames]{color}  % 可以使用顏色
%\usepackage{framed}     % 可以加文字方框
\usepackage{enumerate}  % 加強版enumerate
\usepackage{pgf,tikz}
\usepackage{pgfplots} 
\usepackage{mathrsfs}
\usetikzlibrary{arrows}
\usepackage{datetime}
\usepackage{mathtools}
\usepackage{siunitx}
%%%%%%%%%%%%%%中文 Environment%%%%%%%%%%%%%%%
\usepackage{xeCJK}  % xelatex 中文
\usepackage{multicol} %Multi-columned enumeration
%\usepackage{CJKulem}	% 中文字裝飾
\setCJKmainfont[AutoFakeBold=3,AutoFakeSlant=.4]{Noto Sans CJK TC Light}
\defaultCJKfontfeatures{AutoFakeBold=3,AutoFakeSlant=.4}
%\newCJKfontfamily\Kai{楷體-繁}
%\newCJKfontfamily\Hei{微軟正黑體}
%\newCJKfontfamily\NewMing{新細明體}
%設定中文為系統上的字型,而英文不去更動,使用原TeX字型

\XeTeXlinebreaklocale "zh"
\XeTeXlinebreakskip = 0pt plus 1pt
\usepackage{ruby}
%%%%%%%%%%%%%%%字體大小設定%%%%%%%%%%%%%%%
%\def\normalsize{\fontsize{10}{15}\selectfont}
%\def\large{\fontsize{40}{60}\selectfont}
%\def\Large{\fontsize{50}{75}\selectfont}
%\def\LARGE{\fontsize{90}{20}\selectfont}
%{\arabic{section}}{0em}{}
%\titleformat{\section}{\centering\Large}
%{\arabic{section}}{0em}{}
%\titleformat{\subsection}{\large}
%{\arabic{subsection}}{0em}{}
%\titleformat{\subsubsection}{\bf\normalsize}
% \def\huge{\fontsize{34}{51}\selectfont}
% \def\Huge{\fontsize{38}{57}\selectfont}

%%%%%%%%%%%%%%%Theme Input%%%%%%%%%%%%%%%%
%\input{themes/chapter/neat}

%%%%%%%%%%%titlesec settings%%%%%%%%%%%%%%
%\titleformat{\chapter}{\bf\Huge}
%{\arabic{subsubsection}}{0em}{}
%\titleformat{command}[shape]{format}{label}
%{編號與標題距離}{before}[after]

%%%%%%%%%%%%variable settings%%%%%%%%%%%%%%
%\numberwithin{equation}{section}
%\setcounter{secnumdepth}{4}  %章節標號深度
%\setcounter{tocdepth}{1}  %目錄深度

%%%%%%%%%%%%%%%頁面設定%%%%%%%%%%%%%%%
\setlength{\headheight}{15pt}  %with titling
\setlength{\droptitle}{-1.5cm} %title 與上緣的間距
\parindent=24pt %設定縮排的距離
%\parskip=1ex  %設定行距
%\pagestyle{empty}  % empty: 無頁碼
%\pagestyle{fancy}  % fancy: fancyhdr

%use with fancygdr
%\lhead{\leftmark}
%\chead{}
%\rhead{}
%\lfoot{}
%\cfoot{}
%\rfoot{\thepage}
%\renewcommand{\headrulewidth}{0.4pt}
%\renewcommand{\footrulewidth}{0.4pt}

%\fancypagestyle{firststyle}
%{
%\fancyhf{}
%\fancyfoot[C]{\footnotesize Page \thepage\ of \pageref{LastPage}}
%\renewcommand{\headrule}{\rule{\textwidth}{\headrulewidth}}
%}

%%%%%%%%%%%%%%%重定義一些command%%%%%%%%%%%%%%%
\renewcommand{\contentsname}{目錄}  %設定目錄的標題名稱
\renewcommand{\refname}{參考資料}  %設定參考資料的標題名稱
\renewcommand{\abstractname}{\LARGE Abstract} %設定摘要的標題名稱

%%%%%%%%%%%%%%%特殊功能函數符號設定%%%%%%%%%%%%%%%
%\newcommand{\citet}[1]{\textsuperscript{\cite{#1}}}
\newcommand{\np}[1]{\\[{#1}] \indent}
\newcommand{\transpose}[1]{{#1}^\mathrm{T}}
\newcommand{\adj}{\mathrm{adj}}
%%%% Geometry Symbol %%%%
\newcommand{\degree}{^\circ}
\newcommand{\degre}{\ensuremath{^\circ}}
\newcommand{\Arc}[1]{\wideparen{{#1}}}
\newcommand{\Line}[1]{\overleftrightarrow{{#1}}}
\newcommand{\Ray}[1]{\overrightarrow{{#1}}}
\newcommand{\Segment}[1]{\overline{{#1}}}

%%%% Other Symbol %%%%
\newcommand{\floor}[1]{\lfloor#1\rfloor}
\newcommand{\ceil}[1]{\lceil#1\rceil}
\newcommand{\Cnk}[2]{C_{#2}^{#1}}
\newcommand{\abs}[1]{|#1|}
\newcommand{\divs}{\;|\;}
\newcommand{\divdiv}{\;||\;}
\newcommand{\ndivs}{\!\!\not| \;}
\newcommand{\thm}[1]{{\bf Theorem {#1}}}
\newcommand{\lmm}[1]{{\bf Lemma {#1}}}
\newcommand{\overbar}[1]{\mkern 1.5mu\overline{\mkern-1.5mu#1\mkern-1.5mu}\mkern 1.5mu}
\newcommand{\dd}[2]{\frac{\text{d}#1}{\text{d}#2}}
\newcommand{\pdd}[2]{\frac{\partial#1}{\partial#2}}
\newcommand{\dx}[1]{\,\text{d}#1}
\newcommand{\defeq}{\vcentcolon=}
\newcommand{\sol}[2]{\textbf{Problem #1.} #2}
%%%% OEIS Symbol %%%%
%\newlength{\CapL}
%\AtBeginDocument{\settoheight{\CapL}{A}}
%\newcommand{\oeis}{\includegraphics[height = \CapL]{OEIS.png}}

%%%%%%%%%%%%%%%證明、結論、定義等等的環境%%%%%%%%%%%%%%%
\renewcommand{\proofname}{\bf 證明:} %修改Proof 標頭
\newtheoremstyle{mystyle}% 自定義Style
{6pt}{15pt}%       上下間距
{}%               內文字體
{}%               縮排
{\bf}%            標頭字體
{.}%              標頭後標點
{1em}%            內文與標頭距離
{}%               Theorem head spec (can be left empty, meaning 'normal')

% 改用粗體,預設 remark style 是斜體

\theoremstyle{mystyle}	% 定理環境Style
%
\newtheorem{theorem}{定理}
\newtheorem{definition}{定義}
\newtheorem{formula}{公式}
\newtheorem{condition}{條件}
\newtheorem{supposition}{假設}
\newtheorem{example}{例}
\newtheorem{conclusion}{結論}
\newtheorem{lemma}{引理}
\newtheorem{property}{性質}
\newtheorem{ex}{Example}
\newtheorem{pb}{Problem}
\newtheorem{ppt}{Property}
\newtheorem{df}{Definition}
\newtheorem{thr}{Theorem}
\newtheorem{lm}{Lemma}
\newtheorem{cl}{Corollary}
\newtheorem{hint}{Hint for Problem}
\newtheorem{ans}{Answer for Problem}

\title{YTP Problem 3: 陰晴圓缺 - 解答 / Abundance Sum - Solution}
\author{劉至軒}

\makeatletter %\@title only usable after this
\pagestyle{fancy}
\fancyhf{}
\rhead{劉至軒}
\lhead{\@title}
\rfoot{Page \thepage}

\DeclareMathOperator{\sech}{sech} %Adds \sech function
\DeclareMathOperator{\csch}{csch}

%C++ Code 
\usepackage{xcolor}
\usepackage{minted} 
\usepackage{sourcecodepro}
\usepackage{listings}

\definecolor{bg}{rgb}{0.95,0.95,0.95}
\setminted{fontsize=\footnotesize}

\begin{document}
	\maketitle
	\section*{解答}
		令$N = R - L + 1$為區間長度。
		\subsection*{Subtask 1 - 3/15}
			這個就是暴力解,只需要將 $[L, R]$ 的每一個數字掃過暴力枚舉因數即可。複雜度 $O(N\sqrt{N})$,拿三分。不過因為是基本分數,$R = 10^4$也不是非常大,所以$O(N^2)$也是可以過的哦!
		\subsection*{Subtask 2 - 8/15}
			我們可以假設$L = 1$,因為題目其實就只是叫你求兩個前綴的差而已。這樣子,就可以列式,所求的為 
			
			\begin{align*}
				\sum_{x = 1}^N \Delta(x) =&\sum_{x = 1}^{N} \left(x - \sum_{y < x,\, y \divs x} y\right)\\
				= &\sum_{x = 1}^{N}x -  \sum_{\substack{y < x\\ y \divs x}} y\\
				= &\frac{N(N + 1)}{2} - \sum_{x = 1}^{N} \sum_{\substack{y < x\\ y \divs x}} y
			\end{align*}
			後面的雙重求和比較麻煩,但是其實只需要換一個角度看就很簡單了 —— 與其看 $x$ 求因數,不如看 $y$ 求倍數:對於一個 $y$,它會被幾個 $x$ 數到?其實就是 \textbf{除了他以外的倍數在$N$以內有幾個}那麼多次。也就是說,
			\begin{align*}
				&\frac{N(N + 1)}{2} - \sum_{x = 1}^{N} \sum_{\substack{y < x\\ y \divs x}} y\\
				= &\frac{N(N + 1)}{2} - \sum_{y = 1}^{N} \sum_{\substack{y < x\\y \divs x\\x \in [1, N]}} y\\
				= &\frac{N(N + 1)}{2} - \sum_{y = 1}^{N} \left(\floor{\frac{N}{y} - 1}\right)y\\
				= &N(N + 1) - \sum_{y = 1}^{N} \floor{\frac{N}{y}}y\\
			\end{align*}
			所以對於每一個 $y$都計算一次就好了,複雜度 $O(N)$。
		\subsection*{Subtask 3 - 15/15}			
			我們繼續沿用前面的式子:因為$\floor{\frac{N}{y}}$的值只有 $O(\sqrt{N})$種(事實上,不會超過 $2\sqrt{N}$種),所以用數論分塊可以將$\floor{\frac{N}{y}}$相同的一起算,然後就只是$\floor{\frac{N}{y}}$乘上一些的連續數字相乘,總複雜度 $O(\sqrt{N})$。
	\section*{Solution}
	Let $N = R - L + 1$ be the length of the given interval.
	\subsection*{Subtask 1 - 3/15}
		All that needs to be done is for every number within range, perform a search for all its proper divisors. This solution takes $O(N\sqrt{N})$ time. 
	\subsection*{Subtask 2 - 5/15}
		W.L.O.G. let $L = 1$, since all we are looking for is the difference of two prefix sums. Then what we are looking for is
		\begin{align*}
			\sum_{x = 1}^N \Delta(x) =&\sum_{x = 1}^{N} \left(x - \sum_{y < x,\, y \divs x} y\right)\\
			= &\sum_{x = 1}^{N}x -  \sum_{\substack{y < x\\ y \divs x}} y\\
			= &\frac{N(N + 1)}{2} - \sum_{x = 1}^{N} \sum_{\substack{y < x\\ y \divs x}} y
		\end{align*}
		What's annoying is the double summation at the end. Fortunately, all that's required is a change of perspective: if we not think about divisors but instead of multiples, we can switch the order of summations and sum by $y$ instead. Then for every $y$, it will be counted however many proper multiples under $N$ it has times. So the summation becomes: 
		\begin{align*}
			&\frac{N(N + 1)}{2} - \sum_{x = 1}^{N} \sum_{\substack{y < x\\ y \divs x}} y\\
			= &\frac{N(N + 1)}{2} - \sum_{y = 1}^{N} \sum_{\substack{y < x\\y \divs x\\x \in [1, N]}} y\\
			= &\frac{N(N + 1)}{2} - \sum_{y = 1}^{N} \left(\floor{\frac{N}{y} - 1}\right)y\\
			= &N(N + 1) - \sum_{y = 1}^{N} \floor{\frac{N}{y}}y\\
		\end{align*}
		And so we can just iterate over all $y$ for a linear solution (i.e. runs in $O(N)$).
	\subsection*{Subtask 3 - 15/15}
		All that's needed to get full marks in this problem is just to realise that there are only $O(\sqrt{N})$ different values (actually, no more than $2\sqrt{N}$ distinct values) that $\floor{\frac{N}{y}}$ can take. So we just need to group them up by $\floor{\frac{N}{y}}$ and multiply it by the sum of the integers in that range. This solution runs in $O(\sqrt{N})$ time.
		
	\newpage
	\section*{官解 / AC Code}
	\inputminted[mathescape, linenos,tabsize=4,breaklines,bgcolor=bg]{c++}{./p7_sol.cpp}
		
	
\end{document}
