%mainfile: template.tex
%%%%%%%%%%%%%%%Title的資訊%%%%%%%%%%%%%%%
\title{W9 手寫功課} %標題
\author{劉至軒} %作者
\date{\today} %日期

\thispagestyle{fancy}
\setlength\parindent{24pt}
\setcounter{section}{-1}
	\maketitle
\rhead{W9 手寫功課}
\lfoot{劉至軒}

\pr{
	假設問題是輸入一個數字,再輸出一個數字:$f:\mathbb{R} \mapsto \mathbb{R}^2$能在$O(n^2)$建立一個$n \times n$的表格給$q: \mathbb{R}^2 \mapsto \mathbb{R}$在$2^n$得到答案,然後$g: \mathbb{R} \rightarrow  \mathbb{R}$能在$O(1)$將答案轉換。那時間的$\max$為$2^n$,但是經過$f$之後,$q$的輸入變成$O(n^2)$,所以實際複雜度$O(n^2 + 2^{n^2} + 1) = O(2^{n^2}) \not = 2^n$。
}
\pr{
	因為一個多項式時間函數的輸出一定是多項式的多(如果超過的話,光輸出解答時間就爆掉了!),所以$Q$的輸入一定是多項式,回報給$g$的東西也一定是多項式。甚至,我們可以給界:時間複雜度的次方上界為三個時間複雜度的度數(degree)的和。
}
\pr{
	若$Q$的最佳時間複雜度超過了多項式時間的限制,則用$P$規約為$Q$的解也超過了多項式時間的限制;不過只知道$P$存在一個多項式時間的演算法,不代表一定是用$Q$來解決的,並沒有引起矛盾。所以沒辦法斷言道$Q$一定有多項式時間的演算法。
}
 \pr{
	假設此資料結構存在,則可以$O(n)$將所有值插入,再不斷$(O(1)$查詢極值並刪除,共$O(n)$。這樣,所得到的序列就是原本序列的排序,但是這違反了排序的$O(n \log n)$下界,故不可能。
}
\pr{
	將每一個點做變換:$a_i \mapsto (a_i, a_i)$,再跑一次最近點對得到最近距離$d$。若$d = 0$則回傳"Yes",否則回傳"No"。
}
\pr{
	將序列看成一個圖:點$a_i$連到$a_{a_i}$。因為$1 \leq a_i \leq n-1$,可以知道$a_{a_n} \not= a_n$。則開始找環(用龜兔賽跑法,額外空間$O(1)$):找到了之後,環的入口就是解,因為$a_{a_n}$一定不在環內,一定有至少兩個$a_k$指向環的入口:環的最後一員和剛進環前的元素。而環的起點可以在空間$O(1)$(也是雙指標,一個從龜開始跑,另外一個從$a_n$開始跑,一次跑一個直到撞到就是解答),時間$O(n)$找到。則再退後一步(判是不是即將撞到)輸出即可。(時間$O(n)$、額外空間$O(1)$、沒有動到原本序列)
}