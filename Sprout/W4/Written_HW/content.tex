%mainfile: template.tex
%%%%%%%%%%%%%%%Title的資訊%%%%%%%%%%%%%%%
\title{W4 手寫功課} %標題
\author{劉至軒} %作者
\date{\today} %日期

\thispagestyle{fancy}
\setlength\parindent{24pt}
\setcounter{section}{-1}
	\maketitle
\rhead{W4 手寫功課}
\lfoot{劉至軒}
\pr{
	~\\
	\begin{enumerate}
		\item 總共匹配15次,以下為過程:
		\begin{C++}
mississippi
s
mississippi
 s
mississippi
  s
mississippi
  si
mississippi
   s
mississippi
   si
mississippi
   sip
mississippi
    s
mississippi
     s
mississippi
     si
mississippi
      s
mississippi
      si
mississippi
      sip
mississippi
       s
mississippi
        s
		\end{C++}
	\item 令字元為$c_1, c_2, c_3 \dots, c_{1000}$,則可以證明這個構造方法是對的:
	\begin{align*}
		A &= \underbrace{c_1c_2c_1c_2\dotsc c_2c_1c_2}_{10^6 - (10^3 - 2)\text{個}}c_3c_4c_5c_6\dots c_{1000}\\
		B &= \underbrace{c_1c_2c_1c_2\dotsc c_2c_1c_2}_{10^6 - (10^3 - 2)\text{個}}c_{3}
	\end{align*}
	\end{enumerate}
	易知$B$不在$A$內,且匹配數量$\sum^{10^6 - (10^3 - 2) + 1}_{k = 1} = 499003996506 > 10^9$個
	\item 以上$A$沒有連續的字元,故可以直接用,所以構法同上。
}

\pr{
	~\\
	\begin{enumerate}
	\item 過程:
		\begin{C++}
1 8 5 3 2 6 4 7 
1 8 5 3 
1 8 
1 8 
    5 3 
    3 5 
1 3 5 8 
        2 6 4 7 
        2 6 
        2 6 
            4 7 
            4 7 
        2 4 6 7 
1 2 3 4 5 6 7 8 
		\end{C++}
	\item 過程:
	\begin{C++}
1, 8, 5, 3, 2, 6, 4, 7
1, 3, 2, 4, 8, 5, 6, 7 (4)
1, 2, 3, 4, 8, 5, 6, 7 (3)
1, 2, 3, 4, 5, 6, 8, 7 (6)
1, 2, 3, 4, 5, 6, 7, 8 (7)
	\end{C++}
	\item  過程:
	\begin{C++}
26 15 27 35 17 36 28 16
15 35 26 36 16 27 17 28 
15 16 17 26 27 28 35 36 
	\end{C++}
\end{enumerate}
}

\pr{
	~\\
	\begin{enumerate}
		\item 
		\begin{enumerate}
			\item \inline{merge\_sort}:是stable,因為在合併的時候,兩個值會優先選擇左邊的,可以確保在左邊的值一定會在右邊。
			
			\item  \inline{quick\_sort}:不是stable,因為pivot是隨便選的,到最後他一定在所有同值的最左邊,而其原本不一定就在最左邊, 所以不是stable。
		
			\item
			 \inline{radix\_sort}:是stable,因為如果值遇到同一個,那在左邊的會先被放進去\inline{vector}
			內,會先被放進去\inline{array}裏頭。
		\end{enumerate}
		\item 不能改變排序函式,但是可以改變比較函式:定義一個新的\inline{struct},裡面存了原本的Data和位置,如果遇到Data相同再比較位置即可。若不能定義新的比較函式,那就超載\inline{<}運算子就好了。
	\end{enumerate}
}