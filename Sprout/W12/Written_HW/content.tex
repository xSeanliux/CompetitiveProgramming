
%mainfile: template.tex
%%%%%%%%%%%%%%%Title的資訊%%%%%%%%%%%%%%%
\title{W12 手寫功課} %標題
\author{劉至軒} %作者
\date{\today} %日期

\thispagestyle{fancy}
\setlength\parindent{24pt}
\setcounter{section}{-1}
	\maketitle
\rhead{W12 手寫功課}
\lfoot{劉至軒}

\pr{
	\\~
	\begin{enumerate}
		\item 令四個花色(和Joker)分別為$1$,$10$,$100$⋯⋯$10000$。則對於一張卡$c$,我們定義$H(c) = $花色對應的值加上其數字(對於Joker,第一個是$1$,第二個是$2$)。
		\item 
		還是用程式比較清楚... \inline{n}為目前的根節點,而\inline{n->l}、\inline{n->r}為指向其左右子樹的指標。
		\begin{C++}
string getHash(Node n){
	if(!n) return "0";
	if(n->l && n->r){
		return "4" + getHash(n->l) + getHash(n->r);
	} else if(n->l){
		return "3" + getHash(n->l);
	else if(n->r){
		return "2" + getHash(n->r);
	} else {
		return "1";
	} 
}
		\end{C++}
		則題目中的二元樹的雜湊值分別為:$0, 1, 31, 21, 411$。
	\end{enumerate}
}
\pr{
	\\~
	\begin{enumerate}
		\item 對於每一個字串,都必須要有不同的雜湊值,也就是需要數有幾個相異字串。答案:
		$$\sum_{i = 1}^{6} 26^i$$個不同的值。
		\item 沒辦法,因為可能還需要不知道的資訊,譬如一個Key等,例子:Viginere Cipher中的Key是解碼的必要條件,如果沒有的話只能慢慢猜,非常難回推其他的密文。
		\item 攻擊者可以透過類似Tampermonkey等具有類似功能的工具Intercept到被攻擊者的電腦傳輸到伺服器的雜湊過後的函數,而因為此雜湊不具有$One-wayness$,Intercept之後就可以容易復原原本的密碼/認證機構,而獲得權限。
		\item 令第$i$個輸入為$a_i = i \cdot  1000000007$,則全部都會跑進去餘數為零的Bucket裡面,到最後查詢會變成$O(N)$。
	\end{enumerate}
}
\pr{
	\\~
	\begin{enumerate}
		\item 程式如下:
		\begin{C++}
int M, C;	//given values
int hash(string s){
	int currentPow = 1, res = 0;
	reverse(s.length(), s.end());
	for(char c : s){
		res = (res + (c - 'a') * currentPow) % M;
		currentPow = C * currentPow % M;
	}
	return res;
}
		\end{C++}
		此程式對於每一個字元都跑$O(1)$的運算,所以時間$O(1) \times n = O(n)$。
		\item $s_1 = a$、$s_2 = aa$、$s_3 = aaa$,此處$H(s_1) = H(s_2) = H(s_3) = 0$。
		\item 答案:$$y = C \times (x - s_l \cdot C^{k - 1}) + s_{r + 1} \pmod{M}$$
		\item 先計算$s$的雜湊值$H(s)$,花費時間$O(n)$,然後再對於每一個長度為$n$的$t$的子字串,都可以轉移:先$O(n)$計算$H(t[1, n])$的值,然後轉移到下一個狀態($H[2, n + 1]$),到$H(t[m - n + 1, m])$為$O(1)$,然後在這個步驟總共花了$O(m - n)$。總複雜度$O(n + (n + m - n)) = O(n + m)$。
	\end{enumerate}
	
}