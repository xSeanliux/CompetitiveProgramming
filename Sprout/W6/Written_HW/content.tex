%mainfile: template.tex
%%%%%%%%%%%%%%%Title的資訊%%%%%%%%%%%%%%%
\title{W6 手寫功課} %標題
\author{劉至軒} %作者
\date{\today} %日期

\thispagestyle{fancy}
\setlength\parindent{24pt}
\setcounter{section}{-1}
	\maketitle
\rhead{W6 手寫功課}
\lfoot{劉至軒}

\pr{
	~\\
	\begin{enumerate}
		\item 一個東西插入會是$O(1)$,如果不考慮長度變長的話就是$O(N)$;現在來考慮長度變長的case:會是
		$$\sum^{\lfloor \log N \rfloor}_{k = 0} 2^k = 2^{\lfloor \log N \rfloor + 1} - 1 \approx 2N - 1 = O(N) $$,加起來時間複雜度是$O(N)$。而空間複雜度:$$2^{\ceil{\log N}} \leq 2N = O(N)$$
		\item 顯然的,空間複雜度是$O(N)$,因為\inline{capacity}每次只會加一,所以會是$\min(C_0, N)$,$C_0$是初始的\inline{capacity}。至於時間複雜度,最差的case就是一開始的\inline{capacity}是零,所以每次加都要重新copy一次,複雜度
		$$\sum^n_{k = 0} k = \frac{n(n-1)}{2} = O(n^2)$$
	\end{enumerate}
}
\pr{
	~\\
	\begin{enumerate}
		\item 在每一個\inline{Node}裡面增加一個變數\inline{time},代表進來的時間戳,一開始全部都是\inline{0},一開始時間為\inline{1},每次去\inline{next}就判斷其\inline{time}是否為\inline{0},如果是的話就設定為\inline{++currentTime},否則回傳換的大小為\inline{currentTime - next->time}。這樣,$n$個\inline{Node}都增加了一個變數,所以空間複雜度為$O(n)$。
		\item 可以知道,假設環和起點的距離為$r$,環長度為$\lambda$,則對於$t \geq r$,和所有的$x \in \mathbb{N}$,有
		$$x_t = x_{t + \lambda x}$$,也就是
		$$a \equiv b \pmod{\lambda} \implies x_a = x_b$$
		最差情況就是我跳了$\lambda$步,則顯然$\lambda \equiv 2\lambda \pmod{\lambda}$,而因為$2\lambda - \lambda = \lambda$,所以就直接輸出答案即可。在這之前,都不會有$0 < k < \lambda$使得$x_{k} = x_{2k}$,因為如果$k \equiv 2k \pmod{\lambda} \implies \lambda | (2k - k) = k$,但是$k < \lambda \implies k = 0$,然而$k > 0$,所以不會有。如果$\lambda \leq k$,則就最多跑$k$次,如果有遇到$x_i = x_{2i}$的話,那環的大小就是$i$;否則就輸出沒有環。這樣複雜度就是$O(k)$了。
		\item 令環長為$\lambda = O(N)$,則會經過$\ceil{\log \lambda}$次的尋找才會找到。所以時間複雜度
		$$\sum^{\ceil{\log \lambda}}_{k = 0} 2^k = 2^{\ceil{\log \lambda}+ 1} - 1 = O(\lambda) = O(N)$$
	\end{enumerate}
}
\pr{
	~\\
	\begin{enumerate}
		\item 最爛的case就是每次$a_i$都要跳到$a_{i + 1}$,所以要慢慢跳,跳$O(N)$格,而如果每一個詢問都這樣整人的話,複雜度就是$O(NQ)$了。
		\item 可以表示為$a[a[i]]$。
		\item 應該要填$b[b[i][j-1]][j-1]$。
		\item 令$q = \ceil{\log k}$,那以下的程式會回傳一開始在$x$,跳了$k$步之後的編號:
		\begin{C++}
for(int i = 0; i <= q; i++){
	if((long long)1 << i & k){
		x = b[x][i];	
	}
}
return x;
		\end{C++}
		所以複雜度是$O(\log k)$。
		\item 因為已經假設好$b$陣列已經建立完畢(如果沒有的話可以$O(N \log N)$建立),對於每一個詢問都可以$O(\log k) = O(\log N)$詢問,所以複雜度$O(Q \log N)$。
	\end{enumerate}
}