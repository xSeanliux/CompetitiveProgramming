
%mainfile: template.tex
%%%%%%%%%%%%%%%Title的資訊%%%%%%%%%%%%%%%
\title{W10 手寫功課} %標題
\author{劉至軒} %作者
\date{\today} %日期

\thispagestyle{fancy}
\setlength\parindent{24pt}
\setcounter{section}{-1}
	\maketitle
\rhead{W10 手寫功課}
\lfoot{劉至軒}

\pr{
	~\\
	\begin{enumerate}
		\item $(a, b, c) = (0, 1, 0)$或$(a, b, c) = (1, 0, 1)$
		\item $(a, b, c, d) = (0, 0, 0, 0)$
	\end{enumerate}
}

\pr{
	~\\
	\begin{enumerate}
		\item 
		\begin{align*}
			f &= (c_1 \land c_2 \land \dots \land c_k)\\
			&= (c_1 \land c_2) \land \left[ (c_3 \land c_4 \land \dots \land c_k) \right]\\
			\implies \lnot f &=  \lnot  (c_1 \land c_2) \lor \lnot (c_3 \land c_4 \land \dots \land c_k)  \\
			&= \lnot c_1 \lor \lnot c_2 \lor \lnot \left[(c_3 \land c_4) \land c_5 \dots \land c_k \right]\\
			&= \lnot c_1 \lor \lnot c_2 \lor \lnot c_3 \lor \lnot c_4 \lor \left[(c_5 \land c_6) \land \dots \land c_k \right]\\
			\vdots\\
			&= \lnot c_1 \lor \lnot c_2 \lor \dots \lor \lnot c_k\\
		\end{align*}
		\item
		因為$f$為一個CNF算式,所以型如
		$$f  = c_1 \land c_2 \land \dots \land c_k$$
		則若想要$f$為\inline{F},則至少一個$c_i$為\inline{F},又因為$c_i$中沒有重複變數,所以就一定可以選擇變數,使得$f$是\inline{F}了,這樣子為$O(k)$,$k$為變數數量。
		\item 因為$f$為DNF算式,所以型如
		$$f  = c_1 \lor c_2 \lor \dots \lor c_k$$
		只需要選擇其中一個$c_i$,使得其中的變數經過運算回傳\inline{T}即可,且因為沒有重複的變數所以一定辦得到,複雜$O(k)$,$k$為變數數量。
		\item 同上,$f$為一個DNF算式代表$\lnot f$是一個CNF算式。而判斷$f$是否可以為\inline{F}等同於判斷$\lnot f$是否可以為\inline{T},成為了一個$CNF-SAT$問題。然而,根據Cook-Levin Theorem,$CNF-SAT \in NPC$,又$P \not = NP$,所以其不存在一個多項式演算法。
		\item 直接用構造反證法:
		考慮一個布林算式
		$X = (a_1 \land b_1) \lor (a_2 \land b_2 ) \lor \dots \lor (a_k \land b_k)$
		則這代表「存在某個數字$k$使得$a_k$、$b_k$皆為\inline{T}」,也就代表:「當我每一個數字都選擇$a_i$或$b_i$的時候,不論怎麼選都一定會有一個\inline{T}」,所以CNF樣式為
		$(a_1 \lor a_2 \lor \dots a_k) \land (b_1 \lor a_2 \lor \dots a_k) \land \dots \land (b_1 \lor b_2 \lor \dots b_k) $,長度從原本的$2n$變成$2^n$了, 光輸出就不是多項式了,所以不存在一個多項式時間演算法,因為下界就已經超過了!
	\end{enumerate}
}
\pr{
	令要判斷的第一個為$A$和$B$。若兩者相等,則$A \oplus B$(此處$\oplus$表示位元XOR)恆為\inline{F}。也就是,想要找到一組變數使得$A \oplus B$為\inline{T}即可。又
	$$X = A \oplus B = (\lnot A \land B) \lor (A \land \lnot B)$$
	,若$X$存在一組變數使得回傳\inline{T},則$A$,$B$不等價。已知這個運算式存在一個等價的CNF算式$P$,則問題即化簡為:$P$是否可以回傳\inline{T}?也就是$CNF-SAT$問題,故此問題為$NP-hard$。
}	