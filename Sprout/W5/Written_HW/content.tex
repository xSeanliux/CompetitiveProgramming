%mainfile: template.tex
%%%%%%%%%%%%%%%Title的資訊%%%%%%%%%%%%%%%
\title{W5 手寫功課} %標題
\author{劉至軒} %作者
\date{\today} %日期

\thispagestyle{fancy}
\setlength\parindent{24pt}
\setcounter{section}{-1}
	\maketitle
\rhead{W5 手寫功課}
\lfoot{劉至軒}
\pr{
	~\\
	\begin{enumerate}
		\item 
			假設有一個解是$a_1, a_2, a_3, \cdots a_n$,但是不是最佳解(不是每次都選最大可以拿的),現在我們要證明這個拿的數量一定比最佳解拿的數量多。假設對於其中一步沒有取最大值(不妨設為第一步),那其中一定有一堆可以化成一個更大的幣值的,也就是可以減少錢幣數:要證明的是:如果有許多的小錢幣$x_1, x_2, x_3, \cdots x_k$,其加起來超過$c_i$,則一定可以挑選其中一些個,加起來等於$c_i$,就可以消掉。
			
			\textbf{證明:}
			
				用數學歸納法:假設一個由$c_1 = 1$的錢幣的和大於或等於$c_2$,則答案顯然:選$c_2$個不就得了!那假設一堆比$c_{i}$小的錢幣加起來超過$c_{i}$,且一定可以選擇那些錢幣使得和等於$c_i$。那看$c_{i + 1}$:它有兩種可能:因為$c_i | c_{i +1}$,所以可能全部由$c_{i}$組成,也可能不是。分這兩個case討論:
				\begin{enumerate}
					\item  全部由$c_i$組成,則顯然可以:選$\frac{c_{i + 1}}{c_i}$個就好了!
					\item 假設由$x$($x \leq \frac{c_{i + 1}}{c_i} - 1$) 個$c_i$和其他的,比$c_i$小的錢幣,令其和為$S$。因為$c_i | c_{i + 1}$且$S + x \cdot c_i = c_{i + 1}$所以
					$$c_{i + 1} = S + x \cdot c_i \leq S + c_{i + 1} - c_{i} \implies S \geq c_i$$
					那因為以上的假設,我就可以取一些化簡成$c_i$,而$x$變成$x + 1$。若其依然符合$x \leq \frac{c_{i + 1}}{c_i} - 1$,那就繼續取,直到$x = \frac{c_{i + 1}}{1}$,那就變成第一個case了,就直接取最大的。
				\end{enumerate}
			所以每次如果沒有取最大的,而是取其他的,一定有辦法合併成更好的解,而最後的合併就會變成貪心解,得證。
		\item 令
		$c_1 = 1, c_2 = 3561, c_3 = 3562, x = 7122$,則用貪心法則需要取$3561$個($c_3$一個,$c_1$$3560$個),但是最佳解取兩個($c_2$)就好了,所以貪心法在這個時候不一定會對。
		\item  令$c_1 = 2, c_2 = 3$就會對了:先觀察:發現一定不會取超過$2$個$2$,因為只要超過了,每$3$個$2$就可以換成$2$個$3$,進而減少錢幣數量。對$x$進行模$3$的分析:
			\begin{enumerate}
				\item $x \equiv 0 \pmod{3}$,則$x$為$6k$,$k \in \mathbb{N}$的形式,那一定是全部取3最好。
				\item $x \equiv 1 \pmod{3}$,則$x$為$6k + 1$或$6k + 4$,$k \in \mathbb{N}$的形式,在這兩個case,都是需要拿兩個$2$可以化成$3$的倍數(分別為$6k - 3$和$6k$的形式)
				\item  $x \equiv 2 \pmod{3}$,則則$x$為$6k + 2$或$6k + 5$,$k \in \mathbb{N}$的形式,在這兩個case,都是需要拿一個$2$可以化成$3$的倍數(分別為$6k $和$6k + 3$的形式)
			\end{enumerate}
		由以上分析可以知道貪心法在這個case也會對。
	\end{enumerate}
}

\pr{
	假設我目前選了區間集合$S = \{S_1, S_2, \cdots, S_n\}$,且有
	$$\forall i < j, l_i < r_i < l_j < r_j$$
	則可以一直做這個操作:
	$\forall i$,將$S_i$移動至目前終點最左邊,與之前的線段都不相交的線段。這個操作顯然不會影響到$|S|$(也就是,如果$S_i$做了這個操作,則顯然$S_{i + 1}$一定還是可以存在,因為$S_i$的右界一定會不動或往左,所以$S_{i + 1}$最差就是不動)則到最後,可能有一些剩下在後面的區間。那如果這個$S$剛好是解答呢(令其稱為$S_{ans}$)?對每一個元素都做這個操作之後(變成$S'_{ans}$),$| S'_{ans} | = | S_{ans}  |$維持不變!而且,還有一個好的性質,就是呢,假設$S_{l}$為用貪心法則(排序右界然後取最左邊不相交的)的話,那對於每一個$S_l[i]$都有
	$$S_l[i] = S'_{ans}[i]$$
	總結就是:假設有答案$S_{ans}$,而$S'_{ans}$是對於每一個線段做以上的操作後的集合,則用貪心法取出的集合$S_{l}$會滿足
	$$|S_{ans}| = |S'_{ans}| = |S_l|$$
	所以$S_l$可以得出最多的選法的其中一組解。為什麼呢?因為對於第$i$個線段,其經過運算之後會跑到貪心第$i$步所選到的線段。
}