%mainfile: template.tex
%%%%%%%%%%%%%%%Title的資訊%%%%%%%%%%%%%%%
\title{W2 手寫功課} %標題
\author{劉至軒} %作者
\date{\today} %日期

\thispagestyle{fancy}
\setlength\parindent{24pt}
\setcounter{section}{-1}
	\maketitle
\rhead{W2 手寫功課}
\lfoot{劉至軒}
\pr{
	~\\
\begin{enumerate}
	\item  
	$$\lim_{n \rightarrow \infty} \frac{3n + 1}{n - 1} = \lim_{n \rightarrow \infty} \frac{3 + \frac{1}{n}}{1 - \frac{1}{n}} = 3$$ 
	\item 
	$$\lim_{n \rightarrow \infty} \frac{n}{n^2 + 1} = \lim_{n \rightarrow \infty} \frac{1}{n + \frac{1}{n}} = 0$$
	\item 
	$$f(n) \in O(2^n) \implies \exists \varepsilon, x_0, \forall n > x_0, | f(n) | \leq \varepsilon \cdot 2^{n} = \frac{\varepsilon}{2} \cdot 2^{n+1}$$,所以只要選$\varepsilon' = \varepsilon$即可以符合定義。
	
	反方向:
	$$f(n) \in O(2^{n + 1}) \implies \exists \varepsilon, x_0, \forall n > x_0 | f(n) | \leq \varepsilon \cdot 2^{n + 1} = 2\varepsilon \cdot 2^{n}$$,則選擇$\varepsilon' = 2\varepsilon$則符合定義。
	\item 
	$$f(n) \in O(n!) \implies \exists \varepsilon, x_0, \forall n > x_0, | f(n) | \leq \varepsilon \cdot (n!)  < \varepsilon \cdot (n + 1)!$$
	故命題成立。
	
	反方向:
	$f(n) = (n + 1)!$就是一個反例子:假設
	$$\exists \varepsilon, x_0, \forall n > x_0, |f(n)| \leq \varepsilon \cdot (n)!$$
	再考慮$f(\ceil{\varepsilon})$:
	\begin{align*}
		f(\ceil{\varepsilon}) &= (\ceil{\varepsilon} + 1)!\\
		&= (\ceil{\varepsilon} + 1) \cdot (\ceil{\varepsilon} \rfloor)!\\
		&> \varepsilon \cdot (\varepsilon)!
	\end{align*}
	矛盾。所以此不成立,只有一方面。
	\item 這一題也是舉範例:對於$k > 1$,$f(x) = kx$皆不成立:顯然$f(x) = kx \in O(n)$,只需要取$\varepsilon > k$即可。假設$2^{f(n)} \in O(2^n)$,則
	$$\exists \varepsilon, x_0, \forall n > x_0, | 2^{f(n)} | < \varepsilon \cdot 2^{n}$$
	\begin{align*}
		| 2^{f(n)} | < \varepsilon \cdot 2^{n} \implies 2^{kx} &< \varepsilon \cdot 2^{x}\\
		(2^k)^x&=2^{x + \log_2 \varepsilon}
	\end{align*}
	所以對於任何的$\varepsilon$,當$x > \frac{\log_2 \varepsilon}{k - 1}$則以上不等式不成立,有矛盾,故命題不成立。
\end{enumerate}
}

\pr{
	先計算$m = 1$的case:
	$$f(2^1) = 2f(1) + 4 = 6 \leq 6\log_26$$
	假設$m = k$不等式成立,則考慮$m = k + 1$:

	\begin{align*}
		f(2^{k + 1}) &= 2f(2^k) + 2^{k + 2}\\
		&\leq 2 \cdot 3 \cdot 2^{k} \cdot k + 2^{k + 2}\\
		&\leq (3k + 2) \cdot 2^{k + 1}\\
		&< (3k + 3) \cdot 2^{k + 1}\\
		&= 3 \cdot 2^{k + 1} \cdot \log_2(2^{k + 1})
	\end{align*}
	得證。
}

\pr{
	先驗證$n = 1$的case:
	$$f(1) = 1 \leq 2 \cdot 1^2 - 1 = 1$$
	成立。則假設對於所有$x \leq k$,命題皆成立。現在考慮$n = k + 1$(並且$k = 2l$,$l$為正整數):
	\begin{align*}
		f(k + 1) &= f(\lfloor \frac{k + 1}{2} \rfloor) + f(\ceil{\frac{k + 1}{2}}) + (k + 1)^2\\
			&\leq 2 (\lfloor \frac{k + 1}{2} \rfloor)^2 + 2(\ceil{\frac{k + 1}{2}})^2 + k^2 + 2k + 1\\
			&= 2l^2 + 2(l + 1)^2 + 4l^2 + 4l - 1\\
			&= 8l^2 + 8l + 1\\
			&= 2k^2 + 4k + 1\\
			&= 2(k + 1)^2 - 1
	\end{align*}
	然後再考慮$k = 2l + 1$,$l \in \mathbb{N}$的case:
	\begin{align*}
		f(k + 1) &= f(\lfloor \frac{k + 1}{2} \rfloor) + f(\ceil{\frac{k + 1}{2}}) + (k + 1)^2\\
		&\leq 2(l + 1)^2 + 2(l + 1)^2 + (2l + 2)^2 - 2\\
		&=8(l + 1)^2 - 2\\
		&= 8l^2 + 16l + 6\\
		&< 2(2l + 2)^2 - 1 = 16l^2 + 16l + 7
	\end{align*}
	得證。
	\it{36^{\text{ th}}}
}